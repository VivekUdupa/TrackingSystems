\documentclass[12pt]{article}
\usepackage[left=2cm, right=2cm, top=2cm]{geometry}
\usepackage[utf8]{inputenc} 
\usepackage{graphicx} % to include images
\usepackage{amsmath} % For math mode
\usepackage{caption} % For captions
\usepackage{subcaption} % To use caption while using mini page
\usepackage{amssymb} % To use math symbols
\usepackage{multirow} %To combine multiple rows in a table
\usepackage[table]{xcolor} %To color rows / columns in table
\usepackage{titling} %To vertically center the title page
\usepackage{hyperref} %for URL


%----------------------------MATLAB TEMPLATE -------------------------------------
\usepackage{listings}
\usepackage{color} %red, green, blue, yellow, cyan, magenta, black, white
\definecolor{mygreen}{RGB}{28,172,0} % color values Red, Green, Blue
\definecolor{mylilas}{RGB}{170,55,241}
%-----------------------------------------------------------------------------------------

\title{ECE 8540 \\ Analysis of Tracking Systems \\ 
	Assignment 4}
\author{Vivek Koodli Udupa \\ C12768888}
\date{October - 09, 2018 }

%To make the title page center vertically centered
\renewcommand\maketitlehooka{\null\mbox{}\vfill}
\renewcommand\maketitlehookd{\vfill\null}

\begin{document}

%Displaying Title
\begin{titlepage}
\maketitle
\pagenumbering{gobble}% Remove page numbers (and reset to 1)
\end{titlepage}
\pagenumbering{arabic}% Arabic page numbers (and reset to 1)


%Begin of Report

\section{Introduction}
In this report we consider the problem of filtering; Kalman filtering to be specific. Filtering is a method of mitigating noise in sensor data. A set of sensor data is used to design a model. The model can be used to smooth out the noise and to make prediction about expected future states. However, in the case of tracking problem, the thing that is to be tracked is not expected to follow a linear predictable behavior all the time. If the object did follow a predictable path, then there would be point in tracking it. In such a situation, the behavior of the object could just be modeled perfectly and the future behavior is completely known.  \\
\\
For example, Consider the problem of tracking a bullet fired through some kind of gun. Initially the bullet travels straight, but as time progresses, the bullet is slowed down due to gravity, air resistance and multiple other factors. Thus the bullet follows certain path for a particular duration and then changes its path for the next set period of time. In order to track such an object, that changes behavior slowly but continuously, it is necessary to change the model fit continuously. In the extreme this leads to the concept of updating the model fit after every new sensor reading is recorded. This is how filtering works. \\
\\
In the problem of tracking something, we are trying to find the answer for the question \lq\lq{} where is the object? \rq\rq{}. The answer to this question is rarely if ever certain in real life applications. The answer is mostly a probability distribution of \lq\lq{} the object is likely to be in this area \rq\rq{}. This is because of the uncertainties in the sensor measurement, object's behavior and other noises that might persist. In this report we will consider data that contains dynamic and measurement noises and we will use Kalman filtering to generate predictions. Kalman filtering is an algorithm that uses a set of measurements collected over time, that contains statistical noises and other inaccuracies and produces estimates of unknown variables that tend to be more accurate than those based on single measurement alone, by estimating a joint probability distribution over the variables for each time frame.\\
\\
This report deals with the application of kalman filter to track the position of an object in both 1D and 2D space. This report starts with the visual analysis of dataset, building the kalman filter equations and then finally discussing the results of the derived kalman filter.

\section{Methods}
Kalman filter is a continuous cycle of predict and update. When formulating a problem for the Kalman filter, we consider the following steps: 

\begin{enumerate}
	\item Determine the state variables. Here we are answering the question \lq\lq{}what are we tracking?\rq\rq{}.
	\item Write the state transition equations i.e. How things evolve over time.
	\item Define the dynamic noise(s).  This describes the uncertainties in state transition equation.
	\item Determine the observation variables i.e. Sensor readings.
	\item Write the observation equations (relating the sensor readings to the state variables).
	\item Define the measurement noise(s). These are the uncertainties in observation variables.
	\item Characterize the state transition matrix and observation matrix.
	\item Check all matrices in the Kalman filter equations to make sure the sizes are appropriate.
\end{enumerate} 

\subsection{ Deriving the equations for 1D Model}
\label{sec:1D}
We will begin by describing the things that we are tracking. The state $X_t$ can be defined as: 
\begin{equation}
X_t = 
\begin{bmatrix}
	x_t \\
	\dot{x}_t
\end{bmatrix}
\label{eq:1d state}
\end{equation}
where $x_t$ is the position and $\dot{x}_t$ is the velocity. \\
\\
The second step is to write the state transition equations that describe the expected behavior of the state variables. Here, we are considering a constant 1D velocity model. The state transition equations for the model are as follows: 
\begin{align}
\begin{split}
	X_{t+1} &= x_t + T \dot{x_t} \\
	\\
	\dot{X_t} &=  \dot{x_t}
\end{split}
\label{eq:1d state transition}
\end{align}
\\
The third step is to define the dynamic noise(s). Here we will assume that random accelerations can happen between sensor samples. 
\begin{equation}
\text{dynamic noise} = 
\begin{bmatrix}
	0 \\
	N(0,\sigma^2_{a1})
\end{bmatrix}
\label{eq:1d dyn noise}
\end{equation}
Notice that the first element of the matrix is zero. The zero indicates that there is no noise in the position of the reading. Noise in position reading would indicate that the object is teleporting.  \\
\\
The forth step is to list the observation variables. Here, we are sensing the x position. Denoting the sensor reading for this as follows: 
\begin{equation}
Y_t = 
\begin{bmatrix}
	\tilde{x_t} \\
	0
\end{bmatrix}
\label{eq:1d obs var}
\end{equation}
The second element of the matrix is zero because the velocity is not being measured. \\
\\
The fifth step is to describe the observation equation.
\begin{equation}
\tilde{x_t} = x_t
\label{eq:1d obs eq}
\end{equation}
\\
The sixth step is to define measurement noise(s). Here, we will assume that the sensor reading is corrupted by noise.
\begin{equation}
\text{measurement noise} = 
\begin{bmatrix}
	N(0,\sigma^2_{n1})
\end{bmatrix}
\label{eq:1d mes noise}
\end{equation}
\\
The seventh step is to characterize the co-variance matrices. There are three of them.  \\
The co-variance of the state variables is written as:
\begin{equation}
s_t = 
\begin{bmatrix}
	\sigma^2_{x_t} & \sigma_{x_t,\dot{x_t}} \\
	\sigma_{x_t,\dot{x_t}} & \sigma^2_{\dot{x_t}}
\end{bmatrix}
\label{eq:1d state co-var}
\end{equation}
\\
Co-variance of dynamic noise can be written as:
\begin{equation}
Q = 
\begin{bmatrix}
	0 & 0 \\
	0 & \sigma^2_{a_1}
\end{bmatrix}
\label{eq:1d dyn co-var}
\end{equation}

Co-variance of measurement noise can be written as:
\begin{equation}
R = 
\begin{bmatrix}
	\sigma^2_{n_1} & 0 \\
	0 & 0
\end{bmatrix}
\label{eq:1d mes co-var}
\end{equation}
\\
The eighth step is to define the state transition and observation matrices. The state transition matrix can be obtained from equation \ref{eq:1d state transition}.
\begin{equation}
\phi = 
\begin{bmatrix}
	1 & T \\
	0 & 1
\end{bmatrix}
\label{eq:1d state trans mat}
\end{equation}
\\
The observation matrix can be obtained from equation \ref{eq:1d obs var}.
\begin{equation}
M = 
\begin{bmatrix}
	1 & 0 \\
	0 & 0
\end{bmatrix}
\label{eq:1d state trans mat}
\end{equation}
\\
Note that we have just finished formulating the 1D model for filtering problem. This is just a theoretical model of how we expect the system being tracked to behave. Implementation will be explained in section \ref{sec:implementation}

\subsection{Deriving the equations for 2D Model}
\label{sec:2D}
Similar to the derivation shown in section \ref{sec:1D} this section will show the derivation to a constant 2D velocity model. \\
We will begin by describing the things that we are tracking. The state $X_t$ can be defined as: 
\begin{equation}
X_t = 
\begin{bmatrix}
	x_t \\
	y_t \\
	\dot{x}_t\\
	\dot{y}_t
\end{bmatrix}
\label{eq:2d state}
\end{equation}
where $x_t$ and $y_t$ is the position and $\dot{x}_t$ and $\dot{y}_t$ is the velocity. \\
\\
The second step is to write the state transition equations that describe the expected behavior of the state variables. Here, we are considering a constant 2D velocity model. The state transition equations for the model are as follows: 
\begin{align}
\begin{split}
	X_{t+1} &= x_t + T \dot{x_t} \\
	\\
	Y_{t+1} &= y_t + T \dot{y_t} \\
	\\
	\dot{X_t} &=  \dot{x_t} \\
	\\
	\dot{Y_t} &=  \dot{y_t}
\end{split}
\label{eq:2d state transition}
\end{align}
\\
The third step is to define the dynamic noise(s). Here we will assume that random accelerations can happen between sensor samples. 
\begin{equation}
\text{dynamic noise} = 
\begin{bmatrix}
	0 \\
	0 \\
	N(0,\sigma^2_{a1}) \\
	N(0,\sigma^2_{a2})
\end{bmatrix}
\label{eq:2d dyn noise}
\end{equation}
\\
The forth step is to list the observation variables. Here, we are sensing the x and y positions. Denoting the sensor reading for this as follows: 
\begin{equation}
Y_t = 
\begin{bmatrix}
	\tilde{x_t} \\
	\tilde{y_t}
\end{bmatrix}
\label{eq:2d obs var}
\end{equation}
\\
The fifth step is to describe the observation equation.
\begin{align}
\begin{split}
	\tilde{x_t} = x_t \\
	\tilde{y_t} = y_t
\end{split}
\label{eq:2d obs eq}
\end{align}
\\
The sixth step is to define measurement noise(s). Here, we will assume that the sensor reading is corrupted by noise.
\begin{equation}
\text{measurement noise} = 
\begin{bmatrix}
	N(0,\sigma^2_{n1}) \\
	N(0,\sigma^2_{n2})
\end{bmatrix}
\label{eq:2d mes noise}
\end{equation}
\\
The seventh step is to characterize the co-variance matrices. There are three of them.  \\
The co-variance of the state variables is written as:
\begin{equation}
s_t = 
\begin{bmatrix}
	\sigma^2_{x_t} & \sigma_{x_t,y_t} & \sigma_{x_t,\dot{x_t}} & \sigma_{x_t,\dot{y_t}}  \\
	\sigma_{x_t,y_t} & \sigma^2_{y_t} & \sigma_{y_t,\dot{x_t}} & \sigma_{y_t,\dot{y_t}} \\
	\sigma_{x_t,\dot{x_t}} & \sigma_{y_t,\dot{x_t}} & \sigma^2_{\dot{x_t}} & \sigma_{\dot{x_t},\dot{y_t}} \\
	\sigma_{x_t,\dot{y_t}} & \sigma_{y_t,\dot{y_t}} & \sigma_{\dot{x_t},\dot{y_t}} & \sigma^2_{\dot{y_t}}
\end{bmatrix}
\label{eq:2d state co-var}
\end{equation}
\\
Co-variance of dynamic noise can be written as:
\begin{equation}
Q = 
\begin{bmatrix}
	0 & 0 & 0 & 0\\
	0 & 0 & 0 & 0\\
	0 & 0 & \sigma^2_{a_1} & \sigma_{a_1,a_2} \\
	0 & 0 & \sigma_{a_1,a_2} & \sigma^2_{a_2}
\end{bmatrix}
\label{eq:2d dyn co-var}
\end{equation}
\\
Co-variance of measurement noise can be written as:
\begin{equation}
R = 
\begin{bmatrix}
	\sigma^2_{n_1} & \sigma_{n_1,n_2} \\
	\sigma_{n_1,n_2} & \sigma^2_{n_2} 
\end{bmatrix}
\label{eq:2d mes co-var}
\end{equation}
\\
The eighth step is to define the state transition and observation matrices. The state transition matrix can be obtained from equation \ref{eq:2d state transition}.
\begin{equation}
\phi = 
\begin{bmatrix}
	1 & 0 & T & 0 \\
	0 & 1 & 0 & T \\
	0 & 0 & 1 & 0 \\
	0 & 0 & 0 & 1
\end{bmatrix}
\label{eq:2d state trans mat}
\end{equation}
\\
The observation matrix can be obtained from equation \ref{eq:2d obs var}.
\begin{equation}
M = 
\begin{bmatrix}
	1 & 0 & 0 & 0 \\
	0 & 1 & 0 & 0
\end{bmatrix}
\label{eq:1d state trans mat}
\end{equation}
\\
Note that we have just finished formulating the 2D model for filtering problem. This is just a theoretical model of how we expect the system being tracked to behave. Implementation will be explained in section \ref{sec:implementation}

\subsection{Implementation}
\label{sec:implementation}
As mentioned previously in section \ref{sec:1D} and section \ref{sec:2D} the Kalman filter is a continuous predict and update loop and the equations for Kalman parameters have been derived. This section will describe the implementation of this filter using the equations for the main loop.\\
\\
The equation for predicting the next state is given by:
\begin{equation}
	X_{t,t-1} = \phi\  X_{t-1,t-1}
\label{eq:imp predict}
\end{equation}
Here $X_{t,t-1}$ represents the next state based on the previous state. $\phi$ is the state transition matrix and $X_{t-1,t-1}$ is the previous state.
\\
Predicting the next state co-variance as:
\begin{equation}
	S_{t,t-1} = \phi\  S_{t-1,t-1}\ \phi^T + Q
\label{eq:imp predict state co-var}
\end{equation}
\\
Then obtain the measurements, i.e $Y_t$ \\
\\
Next, calculate the Kalman gain(weights) as:
\begin{equation}
	K_t = S_{t,t-1}\ M^T\ [\ M\ S_{t,t-1}\ M^T + R\ ]^{-1}
\label{eq:imp kalman gain}
\end{equation}
\\
Now using the predicted state, Kalman gain and measurements update the state as:
\begin{equation}
	X_{t,t} = X_{t,t-1} + K_t\ (\ Y_t\ - M\ X_{t,t-1}\ )
\label{eq:imp state update}
\end{equation}
\\
Now update the state co-variance as :
\begin{equation}
	S_{t,t} = [\ I\ -\ K_t\ M\ ]\ S_{t,t-1}
\label{eq:imp state co-var update}
\end{equation}
\\
The implementation of the above described models and equations are done in MATLAB. Please refer the Appendix for the code.  
\section{Results}

\section{Conclusion}



%----------------------------------------------MATLAB LISTING TEMPLATE-------------------------

\lstset{language=Matlab,%
    %basicstyle=\color{red},
    breaklines=true,%
    morekeywords={matlab2tikz},
    keywordstyle=\color{blue},%
    morekeywords=[2]{1}, keywordstyle=[2]{\color{black}},
    identifierstyle=\color{black},%
    stringstyle=\color{mylilas},
    commentstyle=\color{mygreen},%
    showstringspaces=false,%without this there will be a symbol in the places where there is a space
    numbers=left,%
    numberstyle={\tiny \color{black}},% size of the numbers
    numbersep=9pt, % this defines how far the numbers are from the text
    emph=[1]{for,end,break},emphstyle=[1]\color{red}, %some words to emphasise
    %emph=[2]{word1,word2}, emphstyle=[2]{style},    
}


\section{Appendix}

%\subsection{MATLAB Code
%\lstinputlisting{asg1.m}
%\lstinputlisting{find_a.m}
%--------------------------------------------------------------------------------------------------------

\end{document}
